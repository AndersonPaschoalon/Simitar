%\title{Paper Title*\\
%{\footnotesize \textsuperscript{*}Note: Sub-titles are not captured in Xplore and
%should not be used}
%\thanks{Identify applicable funding agency here. If none, delete this.}
%}
\title{Using BIC and AIC for Ethernet traffic model selection. Is it worth?}

\author{\IEEEauthorblockN{1\textsuperscript{st} Anderson dos Santos Paschoalon}
\IEEEauthorblockA{\textit{dept. name of organization (of Aff.)} \\
\textit{State University of Campinas}\\
Campinas, Brazil \\
anderson.paschoalon@gmail.com}
\and
\IEEEauthorblockN{2\textsuperscript{nd} Christian Esteve Rothenberg }
\IEEEauthorblockA{\textit{dept. name of organization (of Aff.)} \\
\textit{State University of Campinas}\\
Campinas, Brazil  \\
chesteve@dca.fee.unicamp.br}
}

\maketitle

\begin{abstract}
In this work, we aim to evaluate how good are the information criteria AIC and BIC inferring which is the best stochastic process to describe Ethernet inter-packet times. Also, we check if there is a practical difference between using AIC or BIC. We use a set of stochastic distributions to represent inter-packet of a traffic trace and calculate AIC and BIC. To test the quality of  BIC and AIC guesses,  we define a cost function based on the comparison of significant stochastic properties for internet traffic modeling, such as correlation, fractal-level and mean. Then, we compare both results.  In this short paper, we present just the results of a public free Skype-application packet capture, but we provide as reference further analyzes on different traffic traces. We conclude that for most cases AIC and BIC can guess right the best fitting according to the standards of Ethernet traffic modeling.
\end{abstract}

\begin{IEEEkeywords}
BIC, AIC, stochastic function, inter-packet times, correlation, Hurst exponent, heavy-tailed distribution, fractal-level, burstiness, linear-regression, weibull, pareto, exponential, normal, poison, maximum likelihood, Ethernet traffic, traffic modeling, fractal-level, pcap file, Skype traffic.
\end{IEEEkeywords}
