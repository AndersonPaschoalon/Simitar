%\title{Paper Title*\\
%{\footnotesize \textsuperscript{*}Note: Sub-titles are not captured in Xplore and
%should not be used}
%\thanks{Identify applicable funding agency here. If none, delete this.}
%}
\title{SIMITAR: realistic and autoconfigurable traffic generation}

\author{\IEEEauthorblockN{1\textsuperscript{st} Anderson dos Santos Paschoalon}
\IEEEauthorblockA{\textit{dept. name of organization (of Aff.)} \\
\textit{State University of Campinas}\\
Campinas, Brazil \\
anderson.paschoalon@gmail.com}
\and
\IEEEauthorblockN{2\textsuperscript{nd} Christian Esteve Rothenberg }
\IEEEauthorblockA{\textit{dept. name of organization (of Aff.)} \\
\textit{State University of Campinas}\\
Campinas, Brazil  \\
chesteve@dca.fee.unicamp.br}
}

\maketitle

\begin{abstract}
It is a well known fact that the type of traffic on network measurements matters. A realistic Ethernet traffic have a different impact over a device compared to constant traffic (such as generated by Iperf), even with the same avarage bandwidth. A burstier traffic can cause more buffers overflows, increace losses and decrease on the measurement acuracy. The number of flows of a traffic may have an impact of flow-oriented nodes, such as SDN switches and controllers. Therefore, in an scenario where software defined networks are going to play an essential role in the future of the networks, a depper validation of new technologies in order to guarantee the SLAs is crucial. In this scenario we come with an anternative for realist traffic generation. Most of open-source realistic traffic generator tools have the modelling layer coupled to the traffic generator API, wich make it difficult to be update to newer lybraries, or require user programming and manual configuration. We propose a sollution called SIMITAR: SnIffing, ModellIng and TrAffic geneRation, wich separetes the processes of data collection, modelling and traffic generation, is flow-oriented, and is autoconfigurable. It creates and use as imputs compact traces descriptos, XML files that model a network trace. Currently we are able to replicate with acuracy some applications traffics and the flow-level properties from the traffic.
\end{abstract}

\begin{IEEEkeywords}
Sniffing, traffic modelling, BIC, AIC, inter-packet times, Wavelet scalling, traffic generator, Burstiness, pcap file, linear regression, Iperf
\end{IEEEkeywords}
